\section{Virtualios mašinos aprašas}

\subsection{Virtualios mašinos samprata}

Virtuali mašina – tai tarsi realios mašinos kopija. Virtualioje mašinoje
surenkame mums reikalingus komponentus, tokius kaip procesorius,
atmintis, įvedimo/išvedimo įrenginiai, suteikiame jiems paprastesnę
vartotojo sąsają. Tuo pačiu palengvinamas programavimo procesas,
nes sudėtingas ar vartotojui nepatogias sąsajas virtualioje mašinoje yra 
aprašomos supaprastintai. Virtuali mašina realizuoja realios mašinos
komandas paprastesniu, lengviau suprantamu būdu interpretuojant virtualios
mašinos komandas kaip realios mašinos komandas ar jų rinkiniu. Taip pat 
virtuali mašina pateikia supaprastintą atminties adresavimą. Visa tai 
leidžia pasiekti realią mašiną ir virtualios mašinos mašininiu kodu 
parašytą programą sėkmingai įvykdyti realioje mašinoje. 

\subsection{Virtualios mašinos komponentų aprašymas}

\begin{description}
  \item[Atmintis] virtualiai mašinai yra paskiriama jos inicijavimo metu,
    o jos dydis nurodytas programos metainformacijoje.
    Ji yra padalinta į dvi dalis – vykdomąjį kodą ir duomenis. Virtuali
    mašina gali keisti tik duomenų dalyje esančią informaciją, o vykdyti
    tik kodo dalyje esančias komandas. Virtuali mašina dirba su 
    8 baitų ilgio žodžiais. Tiek kodo tiek duomenų dalių adresai 
    skaičiuojami nuo 0.
  \item[Procesorius] turi registrus:
    \begin{description}
      \item[IC] Nuorodos. 3 baitų dydžio, skirtas nurodyti vykdomos komandos
        adresą virtualioje atmintyje.
      \item[R1] Duomenų. 8 baitų bendro naudojimo registras.
      \item[R2] Duomenų. 8 baitų bendro naudojimo registras.
      \item[SF] Loginis. 2 baitų registras, skirtas saugoti aritmetinių 
        operacijų 
        loginių (teisinga „1“ arba klaidinga „0“) reikšmių sekai. Lentelėje
        pateikta informacija apie logines reikšmes, bei jas suformuojanti
        \verb|C++| funkcija, kai yra sudedami du skaičiai \verb|sk1| ir
        \verb|sk2|:

        \begin{tabularx}{0.85\textwidth}{|c|c|c|X|} 
          \hline
          Bitas & Trumpinys & Reikšmė & Paaiškinimas \\
          \hline
          0 & CF & pernešimo požymis & įgija „1“ tada, kai sudėties arba
          atimties rezultatas netelpa į žodį \\
          \hline
          6 & ZF & nulio požymis & parodo ar paskutinės operacijos 
          rezultatas yra nulinis \\
          \hline
          7 & SF & ženklo požymis & parodo, koks yra paskutinės operacijos 
          rezultato ženklas („1“ – jei neigiamas) \\
          \hline
          11 & OF & perpildymo požymis & įgija „1“ tada, kai rezultatas
          netelpa skaičių su ženklu diapazone \\
          \hline
        \end{tabularx}
    \end{description}
  \item[Komandų sistema] 
    \begin{description}
        \item[$LR1 xyz$] Atminties ląstelės, kurios adresas 
          $(x \cdot 10 + y) \cdot 10 + z)$, turinio kopijavimas į 
          registrą $R1 (R1 := [(x \cdot 10 + y) \cdot 10 + z])$.
        \item[$LR2 xyz$] Atminties ląstelės, kurios adresas 
          $(x \cdot 10 + y) \cdot 10 + z$, turinio kopijavimas į 
          registrą $R2 (R2:=[(x \cdot 10+y) \cdot 10+z])$.
        \item[$SR1 xyz$] Registro $R1$ turinio kopijavimas į atminties 
          ląstelę, kurios adresas 
          $(x \cdot 10+y) \cdot 10+z ([(x \cdot 10+y) \cdot 10+z]:=R1)$.
        \item[$SR2 xyz$] Registro $R2$ turinio kopijavimas į atminties 
          ląstelę, kurios adresas 
          $(x \cdot 10+y) \cdot 10+z ([(x \cdot 10+y) \cdot 10+z]:=R2)$.
        \item[$ADD$] Prie registro $R1$ reikšmės pridedama registro $R2$ 
          reikšmė. Formuoja $SF$ požymius. $(R1:=R1+R2)$.
        \item[$ADD xyz$] Sudedamos registrų $R1$ ir $R2$ reikšmės bei 
          rezultatas išsaugojamas atminties ląstelėje, kurios adresas 
          $(x \cdot 10+y) \cdot 10+z$. Formuoja $SF$ požymius. 
          $([(x \cdot 10+y) \cdot 10+z]:=R1+R2)$.
        \item[$SUB$] Iš registro $R1$ reikšmės atimama registro $R1$ 
          reikšmė. Formuoja $SF$ požymius. $(R1:=R1-R2)$.
        \item[$SUB xyz$] Iš registro $R1$ atimama registro $R2$ reikšmė ir 
          rezultatas įrašomas į atminties ląstelę, kurios adresas 
          $(x \cdot 10+y) \cdot 10+z$. Formuoja $SF$ požymius. 
          $([(x \cdot 10+y) \cdot 10+z]:=R1-R2)$.
        \item[$DIV$] Registro $R1$ reikšmė padalinama iš registro $R2$ 
          reikšmės ir dalmuo įrašomas į registrą $R1$, o liekana - į 
          registrą $R2$. Formuoja $SF$ požymius.
          %FIXME: reikia trečio akumuliatoriaus registro arba steko, kad 
          %realizuoti dalybą!
          % Taigi mūsų procesorius „moka“ dalinti…
        \item[$CMP$] Palygina registrus $R1$ ir $R2$. Formuoja $SF$ požymius
          ($R1>R2$, tai $ZF=0$ ir $SF=0; R1=R2$, tai $ZF=1; R1<R2$, 
          tai $ZF=0$ ir $SF=1$).
        \item[$JMP$] Nesąlyginis valdymo perdavimas kodo segmento žodžiui
          adresu 
          $(10 \cdot x+y) \cdot 10+z (IC:=(10 \cdot x+y) \cdot 10+z)$.
          % Kaip suprasti šitą sakinį? Valdymas žodžiui?
        \item[$JE xyz$] Jei $ZF=0$, tai valdymas perduodamas kodo segmento 
          žodžiui adresu $(10 \cdot x+y) \cdot 10+z$.
        \item[$JG xyz$] Jei $ZF=0$ ir $SF=0$, tai valdymas perduodamas kodo
          segmento žodžiui adresu $(10 \cdot x+y) \cdot 10+z$.
        \item[$JB xyz$] Jei $SF=1$, tai valdymas perduodamas kodo segmento
          žodžiui adresu $(10 \cdot x+y) \cdot 10+z$.
        \item[$PD xyz$] Išsiunčia 2 žodžius (pradedant adresu 
          $(10 \cdot x+y) \cdot 10+z)$ į išvesties įrenginį.
          % FIXME Bendraujama įrašais. Žr. RM dokumentaciją.
        \item[$GD xyz$] Į 2 žodžius (pradedant adresu 
          $(10 \cdot x+y) \cdot 10+z)$ įrašoma informacija gauta iš 
          įvesties įrenginio.
          % FIXME Bendraujama įrašais. Žr. RM dokumentaciją.
        \item[$FO xyz$] Atidaromas failas \verb|(FIXME)|%FIXME:
        \item[$FGD xyz$] Skaityti iš failo žodį ir jo reikšmė įrašyti į
          atminties ląstelę adresu $(10 \cdot x+y) \cdot 10+z$.
          % FIXME Bendraujama įrašais. Žr. RM dokumentaciją.
        \item[$FPD xyz$] Skaityti iš failo vieną žodį ir įrašyti jo 
          reikšmę į ląstelę, esančią adresu $(10 \cdot x+y) \cdot 10+z$.
        \item[$FC xyz$] Uždaromas failas \verb|(FIXME)|%FIXME:
        \item[$FD xyz$] Pašalinti failą \verb|(FIXME)|%FIXME:
        \item[$HALT$] Užbaigti programos darbą.`
    \end{description}
  \end{description} 
  
  %XXX Jei procesas veikia supervizoriaus rėžimu, tai jis yra VM ar nėra?
  % Tai yra ar egzistuoja tokia VM, kuri gali tiesiogiai bendrauti su
  % įvedimo ir išvedimo įrenginiais?
   %TODO:
   %2.3) Virtualios mašinos bendravimo su įvedimo/išvedimo įrenginiais
   %mechanizmo aprašymas.
\subsection{Virtualios mašinos bendravimo su įvedimo/išvedimo įrenginiais}

   %TODO:
   %2.4) Virtualios mašinos interpretuojamojo ar kompiliuojamo vykdomojo
   %failo išeities teksto formatas. Pavyzdžiui, kaip išskiriamas duomenų
   %segmentas, kodo segmentas, kaip aprašomi duomenys ir t.t.)
\subsection{Virtualios mašinos interpretuojamojo ar kompiliuojamo vykdomojo 
failo išeities teksto formatas}

   %TODO:
   %2.5) Modeliuojamos virtualios mašinos loginių komponentų sąryšio su
   %realios mašinos techninės įrangos komponentais aprašymas.
\subsection{Modeliuojamos virtualios mašinos loginių komponentų sąryšio su 
realios mašinos techninės įrangos komponentais aprašymas}
   
%Visi skaičiai yra su ženklu!

%Komandų argumentai nurodyti 16-ainiais skaičiais.
