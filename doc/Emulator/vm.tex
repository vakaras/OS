\section{Virtualios mašinos aprašas}

\subsection{Virtualios mašinos samprata}

Virtuali mašina - tai tarsi realios mašinos kopija. Virtualioje mašinoje
surenkame mums reikalingus komponentus, tokius kaip procesorius,
atmintis, įvedimo/išvedimo įrenginiai, suteikiame jiems paprastesnę
vartotojo sąsają. Tuo pačiu palengvinamas programavimo procesas,
nes sudėtingas ar vartotojui nepatogias sąsajas virtualioje mašinoje yra 
aprašomos supaprastintai. Virtuali mašina realizuoja realios mašinos
komandas paprastesniu, lengviau suprantamu būdu interpretuojant virtualios
mašinos komandas kaip realios mašinos komandas ar jų rinkiniu. Taip pat 
virtuali mašina pateikia supaprastintą atminties adresavimą. Visa tai 
leidžia pasiekti realią mašiną ir virtualios mašinos mašininiu kodu 
parašytą programą sėkmingai įvykdyti realioje mašinoje. 

